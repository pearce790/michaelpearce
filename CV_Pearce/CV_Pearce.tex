%%%%%%%%%%%%%%%%%%%%%%%%%%%%%%%%%%%%%%%%%
% Medium Length Professional CV
% LaTeX Template
% Version 2.0 (8/5/13)
%
% This template has been downloaded from:
% http://www.LaTeXTemplates.com
%
% Original author:
% Rishi Shah 
%
% Important note:
% This template requires the resume.cls file to be in the same directory as the
% .tex file. The resume.cls file provides the resume style used for structuring the
% document.
%
%%%%%%%%%%%%%%%%%%%%%%%%%%%%%%%%%%%%%%%%%

%----------------------------------------------------------------------------------------
%	PACKAGES AND OTHER DOCUMENT CONFIGURATIONS
%----------------------------------------------------------------------------------------

\documentclass{resume} % Use the custom resume.cls style

\usepackage[left=0.7in,top=0.6in,right=0.7in,bottom=0.6in]{geometry}
\usepackage{scrextend}
\usepackage{url}
\newcommand{\tab}[1]{\hspace{.2667\textwidth}\rlap{#1}}
\newcommand{\itab}[1]{\hspace{0em}\rlap{#1}}
\name{Michael Pearce} % Your name
\address{Reed College, Department of Mathematics and Statistics} % Your address
\address{3203 SE Woodstock Blvd, Portland, OR 97202} % Your secondary addesss
\address{michaelpearce@reed.edu} % Your phone number and email

\begin{document}

%----------------------------------------------------------------------------------------
%	EDUCATION SECTION
%----------------------------------------------------------------------------------------

\begin{rSection}{Appointments}

\textbf{Assistant Professor of Statistics}
\\ Reed College, Department of Mathematics and Statistics \hfill{2023 - Present}
\end{rSection}

\begin{rSection}{Education}

\textbf{University of Washington}, Seattle, WA \hfill {2018 - 2023} 
\\ Ph.D., Statistics ({\it Statistics in the Social Sciences track})
\\ Advisor: Elena A. Erosheva
\\
\\\textbf{St. Olaf College}, Northfield, MN \hfill {2013 - 2017} 
\\ B.A., Mathematics ({\it Statistics concentration})
\\ {\it Summa cum laude, Phi Beta Kappa}
\end{rSection}


\begin{rSection}{Scholarly Publications}

Gallo, S.A., \textbf{Pearce, M.}, Lee, C.J., and Erosheva, E.A. (2023) ``A new approach to peer review assessments: Score, then rank." \textit{Research Integrity and Peer Review} 8.10: 10.

\textbf{Pearce, M.} and Erosheva, E.A. (2023+) ``Modeling preferences: A Bayesian mixture of finite mixtures for rankings and ratings" \textit{arXiv preprint arXiv:2301.09755} (\textit{under revision})

Perlman, M. and \textbf{Pearce, M.} (2023+) ``Estimating the ratio of means in a zero-inflated Poisson mixture model." (\textit{under revision})

\textbf{Pearce, M.} and Erosheva, E.A. (2022) ``On the validity of bootstrap uncertainty estimates in the Mallows-Binomial model." \textit{arXiv preprint arXiv:2206.12365}.

\textbf{Pearce, M.} and Erosheva, E.A. (2022) ``A unified statistical learning model for rankings and scores with application to grant panel review." \textit{Journal of Machine Learning Research} 23.210: 1--33.

\textbf{Pearce, M.} and Raftery, A.E. (2021) ``Probabilistic forecasting of maximum human lifespan by 2100 using Bayesian population projections." {\em Demographic Research} 44.52: 1271--1294.

\textbf{Pearce, M.}$^*$, Sparrow, Z.$^*$, Mabote, T. R., and Sanchez-Gonzalez, R. (2020) ``stoBEST: An efficient methodology for increased spatial resolution in two-component molecular tagging velocimetry." {\em Measurement Science and Technology} 32.3: 035302

{\em $^*$indicates authors contributed equally.}
\end{rSection}

\begin{rSection}{Other Publications}

\textbf{Pearce, M.} and Raftery, A.E. (2021) ``Will this be a record-breaking century for human longevity?" {\em Significance}.

\textbf{Pearce, M.} and Raftery, A.E. (2021) ``The maximum human life span will likely increase this century, but not by more than a decade" {\em The Conversation}.

\end{rSection}


\begin{rSection}{Teaching and Mentorship}

\textbf{Reed College}
\\ Instructor, MATH 141 (Introduction to Probability and Statistics) \hfill {Fall 2023}

\textbf{University of Washington}
\\ Instructor, CSSS 508 (Introduction to R for Social Scientists) \hfill {Autumn 2022, Spring 2023}
\\ Teaching Assistant, STAT 528 (Applied Statistics Capstone) \hfill {Winter 2021, Winter 2022}
\\ Teaching Assistant, CSSS 589 (Multivariate Data Analysis for the Social Sciences) \hfill Autumn 2021
\\ Teaching Assistant, STAT 498/CSSS 594 (Statistics and Philosophy of Voting) \hfill Autumn 2020
\\ Teaching Assistant, STAT 311 (Elements of Statistical Methods) \hfill {Autumn 2018, Winter 2019}
\\ Teaching Assistant, STAT 342 (Introduction to Probability and Mathematical Statistics) \hfill {Spring 2019}
\\ Teaching Assistant, STAT 220 (Statistical Reasoning) \hfill {Autumn 2019}
\\ Directed Reading Program Mentor, ``Social Choice Analysis of Peer Review Data" \hfill {Spring 2022}
\\ Directed Reading Program Mentor, ``Voting, Ranking, and Preference Modeling" \hfill {Autumn 2021}
\\ Directed Reading Program Mentor, ``Nonlinear Regression" \hfill {Winter 2020, Winter 2021, Spring 2021}
\\ Directed Reading Program Mentor, ``History and Practice of Data Communication" \hfill {Autumn 2020}
\\ Washington eXperimental Mathematics Lab Mentor, ``Improving Panel Consensus Tool" \hfill {Autumn 2021}

\textbf{St. Olaf College}
\\ Supplemental Intructor, MATH 126 (Calculus II) \hfill {Spring 2017}
\\ Supplemental Instructor, MATH 242 (Modern Computational Mathematics) \hfill {Spring 2017}
\\ Student Mentor, TRiO Upward Bound Program \hfill {2013 - 2015}

\end{rSection}


\begin{rSection}{Selected Talks and Presentations}

\textbf{Statistics and Machine Learning in the Social Sciences Working Group} \hfill{November 2023}\\
University of Washington, Seattle, WA \\
``Rankings and ratings in peer review: A mixture of finite mixtures accounting for degree of reviewer leniency" (invited presentation)

\textbf{International Meeting of the Psychometrics Society}, College Park, MD \hfill{July 2023}\\
``Improving preference analysis: Joint models for ordinal and cardinal data" (presentation)

\textbf{NeurIPS}, New Orleans, LA \hfill {December 2022}\\
``A Unified Statistical Learning Model for Rankings and Scores with Application to Grant Panel Review" (Journal-to-Conference Track Poster Session)

\textbf{Joint Statistical Meetings}, Washington, D.C. \hfill {August 2022}\\
``Using ranking data for decision-making" (topic-contributed paper session; organizer and chair)\\
``Fast Bayesian estimation for ranking models" (speed session)

\textbf{ISBA World Meeting}, Montreal, Canada \hfill {June 2022}\\
``Joint Bayesian inference for rankings and ratings under heterogeneous preferences" (poster session)

\textbf{Working Group on Model-Based Clustering}, Athens, Greece (virtual)  \hfill {October 2021}\\
``Unified latent class modeling of scores and rankings applied to grant panel review" (poster session)

\textbf{Joint Statistical Meetings}, Seattle, WA (virtual) \hfill {August 2021}\\
``Unified latent class modeling of score and rank data applied to grant panel review" (speed session)

\end{rSection}


\begin{rSection}{Selected Media Coverage}


\textbf{BBC News (Brazil)} ``Por que cada vez mais pessoas est\~{a}o vivendo at\'{e} os 100 anos?" (July 11, 2022)

\textbf{Stats and Stories (Podcast)} ``The Age of the Supercentenarian" (April 29, 2022) \url{https://statsandstories.net/health1/the-age-of-the-supercentenarian}.

\textbf{Washington Post} ``Want to add healthy years to your life? Here’s what new longevity research says." (Oct. 11, 2021)

\textbf{Southern Weekly (China)} ``What is the limit of human life span?" (Sept. 16, 2021)

\textbf{CNBC} ``Researchers say the probability of living past 110 is on the rise — here’s what you can do to get there" (July 17, 2021)

\textbf{Elemental (Medium)} ``How Long Can Humans Really Live?" (July 15, 2021)

\textbf{Gulf News} ``Surviving up to 150: How long can a person live?" (July 12, 2021)

\textbf{Indian Express} ``Can a person live to age 124, 135 or 150? Some optimism, some caveats" (July 6, 2021)

\textbf{The South African} ``Rise of the supercentenarians: Today’s kids could live for 130 years" (July 4, 2021)

\textbf{UW News} ``How long can a person live? The 21st century may see a record-breaker" (July 1, 2021)


\end{rSection}



\begin{rSection}{Professional Experience}

\textbf{Boeing Research and Technology}\hfill{2019 - 2020}
\\{\it Applied Statistics Intern}

\textbf{Deloitte LLC}\hfill{2017 - 2018}
\\{\it Analytics Consultant}

\end{rSection}

\begin{rSection}{Software}

\textbf{rankrate: Statistical Tools for Preference Learning with Rankings and Ratings.} R package available on CRAN. Vignettes: \url{https://pearce790.github.io/rankrate/} 

\textbf{Peer Review with Rankings and Ratings.} R Shiny application. \url{https://pearce790.shinyapps.io/rankrate_PeerReview/}

\end{rSection}


\begin{rSection}{Honors and Awards}

\textbf{Scholar Award} {\it (NeurIPS)} \hfill {2022}
\\ \textbf{Dorothy M. Gilford Teaching Award} {\it (University of Washington)} \hfill {2021}


\end{rSection}


\begin{rSection}{Academic Service and Outreach}

\textbf{Reed College}
\\ Statistics Tenure-Track Search Committee \hfill{2023}
\\ Undergraduate Research Committee \hfill{2023-2024}

\textbf{University of Washington}
\\ Pre-Application Review Service (reviewer) \hfill{2022}
\\ PhD Admissions (reviewer) \hfill{2020-2021}
\\ International Conference on Machine Learning (workflow chair) \hfill{2021}
\\ Queer Union for (Bio)statistician Inclusion and Community (founder) \hfill{2022-2023}
\\ Undergraduate Curriculum Revamp Committee (member) \hfill {2021-2022}

\end{rSection}


\end{document}
