%----------------------------------------------------------------------------------------
%	PACKAGES AND OTHER DOCUMENT CONFIGURATIONS
%----------------------------------------------------------------------------------------

\documentclass{resume} % Use the custom resume.cls style

\usepackage[left=0.7in,top=0.6in,right=0.7in,bottom=0.6in]{geometry}
\usepackage{scrextend}
\usepackage{hyperref}
\usepackage{url}
\newcommand{\tab}[1]{\hspace{.2667\textwidth}\rlap{#1}}
\newcommand{\itab}[1]{\hspace{0em}\rlap{#1}}
\name{Michael Pearce} % Your name
\address{Reed College, Department of Mathematics and Statistics} % Your address
\address{3203 SE Woodstock Blvd, Portland, OR 97202} % Your secondary addesss
\address{https://pearce790.github.io/ \big | michaelpearce@reed.edu} % Your phone number and email

\begin{document}

\begin{rSection}{Current Appointment}

\textbf{Assistant Professor of Statistics (tenure-track)}
\\ Reed College, Department of Mathematics and Statistics \hfill{2023--Present}
\end{rSection}

\begin{rSection}{Education}

\textbf{University of Washington}, Seattle, WA \hfill {2018--2023} 
\\ Ph.D., Statistics ({\it Statistics in the Social Sciences} track)
\\ Advisor: Elena A. Erosheva
\\ Dissertation: Methods for the Statistical Analysis of Preferences, with Applications to Social Science Data
\\
\\\textbf{St. Olaf College}, Northfield, MN \hfill {2013--2017} 
\\ B.A., Mathematics (Statistics concentration)
\\ {\it Summa cum laude, Phi Beta Kappa}
\end{rSection}


\begin{rSection}{Scholarly Publications}

\textbf{Pearce, M.} and Erosheva, E.A. (2024+) ``Bayesian Rank-Clustering" \textit{arXiv preprint arXiv:2406.19563} (\textit{submitted})

\textbf{Pearce, M.} and Erosheva, E.A. (2024+) ``Modeling preferences: A Bayesian mixture of finite mixtures for rankings and ratings" \textit{arXiv preprint arXiv:2301.09755} (\textit{in revision})

\textbf{Pearce, M.} and Perlman, M. (2024+) ``Estimating the ratio of means in a zero-inflated Poisson mixture model." (\textit{in revision})

Gallo, S.A., \textbf{Pearce, M.}, Lee, C.J., and Erosheva, E.A. (2023) ``A new approach to peer review assessments: Score, then rank." \textit{Research Integrity and Peer Review} 8.10: 10.

\textbf{Pearce, M.} and Erosheva, E.A. (2022) ``On the validity of bootstrap uncertainty estimates in the Mallows-Binomial model." \textit{arXiv preprint arXiv:2206.12365}.

\textbf{Pearce, M.} and Erosheva, E.A. (2022) ``A unified statistical learning model for rankings and scores with application to grant panel review." \textit{Journal of Machine Learning Research} 23.210: 1--33.

\textbf{Pearce, M.} and Raftery, A.E. (2021) ``Probabilistic forecasting of maximum human lifespan by 2100 using Bayesian population projections." {\em Demographic Research} 44.52: 1271--1294.

\textbf{Pearce, M.}$^*$, Sparrow, Z.$^*$, Mabote, T. R., and Sanchez-Gonzalez, R. (2020) ``stoBEST: An efficient methodology for increased spatial resolution in two-component molecular tagging velocimetry." {\em Measurement Science and Technology} 32.3: 035302

{\em $^*$indicates authors contributed equally.}
\end{rSection}

\begin{rSection}{Other Publications}

\textbf{Pearce, M.} and Raftery, A.E. (2021) ``Will this be a record-breaking century for human longevity?" {\em Significance}.

\textbf{Pearce, M.} and Raftery, A.E. (2021) ``The maximum human life span will likely increase this century, but not by more than a decade" {\em The Conversation}.

\end{rSection}

\begin{rSection}{Scholarly Presentations}

\begin{enumerate}
\item ``Statistical Estimation with Ranked Choice Voting" \textit{University of Washington}, Seattle, WA, October 2024. (\textit{Invited lecture in STAT 452})
\item ``Bayesian Rank-Clustering" \textit{Washington State University Vancouver}, Vancouver, WA, October 2024. (\textit{Invited seminar})
\item ``Bayesian Rank-Clustering" \textit{Penn State University; Health Science Department}, University Park, PA, October 2024. (\textit{Invited seminar; virtual})
\item ``Bayesian Rank-Clustering" \textit{Reed College}, Portland, OR, September 2024. (\textit{Math and Statistics department colloquium})
\item ``Broadening Access to Bayesian Statistics with Active Learning" \textit{Joint Statistical Meetings}, Portland, OR, August 2024. (\textit{Invited})
\item ``Bayesian Rank Clustering" \textit{ISBA World Meeting,} Venice, IT, July 2024. (\textit{Contributed})

\item ``Methods for the Statistical Analysis of Preferences, with Applications to Social Science Data" \textit{Classification Society Annual Meeting}, Kelowna, BC, June 2024. (\textit{Invited; Distinguished Dissertation Award session}).
\item ``Bayesian Rank Clustering" \textit{Center for Statistics in the Social Sciences 25th Anniversary Conference} Seattle, WA, May 2024. (\textit{Contributed})
\item ``Weighted Mallows-Binomial Model for Rankings and Ratings" \textit{Statistics and Machine Learning in the Social Sciences Working Group}, Seattle, WA, April 2024. (\textit{Invited})
\item ``Rankings and ratings in peer review: A mixture of finite mixtures accounting for degree of reviewer leniency" \textit{Statistics and Machine Learning in the Social Sciences Working Group}, Seattle, WA, November 2023. (\textit{Invited})
\item ``Improving preference analysis: Joint models for ordinal and cardinal data" \textit{International Meeting of the Psychometrics Society}, College Park, MD, July 2023. (\textit{Contributed})
\item ``A Unified Statistical Learning Model for Rankings and Scores with Application to Grant Panel Review" \textit{NeurIPS}, New Orleans, LA, December 2022. (\textit{Contributed})
\item ``Fast Bayesian estimation for ranking models" \textit{Joint Statistical Meetings}, Washington, DC, August 2022. (\textit{Contributed})
\item ``Using ranking data for decision-making" \textit{Joint Statistical Meetings}, Washington, DC, August 2022. (\textit{Session Chair})
\item ``Joint Bayesian inference for rankings and ratings under heterogeneous preferences" \textit{ISBA World Meeting,} Montreal, QC, June 2022. (\textit{Contributed})
\item ``Unified latent class modeling of scores and rankings applied to grant panel review" \textit{Working Group on Model-Based Clustering}, Athens, Greece, October 2021. (\textit{Contributed; virtual})
\item ``Unified latent class modeling of score and rank data applied to grant panel review" \textit{Joint Statistical Meetings}, Seattle, WA, August 2021. (\textit{Contributed; virtual})
\end{enumerate}

\end{rSection}

\begin{rSection}{Selected Media Coverage}

\begin{itemize}
\item \textit{BBC News (Brazil)} ``Por que cada vez mais pessoas est\~{a}o vivendo at\'{e} os 100 anos?" (July 11, 2022)
\item \textit{Stats and Stories (Podcast)} ``The Age of the Supercentenarian" (April 29, 2022) \url{https://statsandstories.net/health1/the-age-of-the-supercentenarian}.
\item \textit{Washington Post} ``Want to add healthy years to your life? Here’s what new longevity research says." (Oct. 11, 2021)
\item \textit{Southern Weekly (China)} ``What is the limit of human life span?" (Sept. 16, 2021)
\item \textit{CNBC} ``Researchers say the probability of living past 110 is on the rise — here’s what you can do to get there" (July 17, 2021)
\item \textit{Elemental (Medium)} ``How Long Can Humans Really Live?" (July 15, 2021)
\item \textit{Gulf News} ``Surviving up to 150: How long can a person live?" (July 12, 2021)
\item \textit{Indian Express} ``Can a person live to age 124, 135 or 150? Some optimism, some caveats" (July 6, 2021)
\item \textit{The South African} ``Rise of the supercentenarians: Today’s kids could live for 130 years" (July 4, 2021)
\item \textit{UW News} ``How long can a person live? The 21st century may see a record-breaker" (July 1, 2021)
\end{itemize}

\end{rSection}

\newpage
\begin{rSection}{Teaching}

\textbf{Reed College}
\\ Instructor, MATH 141 (Introduction to Probability and Statistics) \hfill {Fall 2023, Fall 2024}
\\ Instructor, MATH 346 (Bayesian Statistics) \hfill {Spring 2024}

\textbf{University of Washington}
\\ Instructor, CSSS 508 (Introduction to R for Social Scientists) \hfill {Autumn 2022, Spring 2023}
\\ Teaching Assistant, CSSS 589 (Multivariate Data Analysis for the Social Sciences) \hfill Autumn 2021
\\ Teaching Assistant, CSSS 594 (Statistics and Philosophy of Voting) \hfill Autumn 2020
\\ Teaching Assistant, STAT 220 (Statistical Reasoning) \hfill {Autumn 2019}
\\ Teaching Assistant, STAT 311 (Elements of Statistical Methods) \hfill {Autumn 2018, Winter 2019}
\\ Teaching Assistant, STAT 342 (Introduction to Probability and Mathematical Statistics) \hfill {Spring 2019}
\\ Teaching Assistant, STAT 528 (Applied Statistics Capstone) \hfill {Winter 2021, Winter 2022}
\\ Mentor, Directed Reading Program \hfill{7 quarters; Autumn 2020--Spring 2022}
\\ Mentor, Washington eXperimental Mathematics Lab \hfill {Autumn 2021}

\end{rSection}

\begin{rSection}{Thesis Mentoring}
Every student at Reed College writes a senior thesis. Below is a list of students whom I have advised.
\begin{itemize}
\item Conor Bekaert, ``Towards a Weighted Joint Statistical Model for Rankings and Ratings" (2023-24)
\item Quinn Hargrove, ``Visualizations to Improve Ranked Data Analyses, with Applications to Board Game Data" (2023-24)
\end{itemize}

\end{rSection}

\begin{rSection}{Service}

\textbf{Reed College}
\begin{itemize}
\item Co-organizer, Statistics tenure-track search informational session (2024)
\item Organizer, Math and Statistics department colloquium (2023-24)
\item Co-organizer, Math and Statistics graduate school panel (2023, 2024)
\item Reviewer, Summer scholarships in math and statistics (2024)
\item Panelist, Math and Statistics department admissions event (2024)
\item Organizer, Statistics curriculum retreat (2024)
\end{itemize}

\textit{College-wide committees:}
\begin{itemize}
\item Statistics tenure-track search (F2023, S2024, F2024)
\item Statistics visiting search (S2024)
\item Library Board (2024--25)
\item Undergraduate Research Committee (2023--24)
\end{itemize}

\textbf{University of Washington}
\begin{itemize}
\item Reviewer, Pre-application review service (2022)
\item Reviewer, PhD program admission (2020, 2021)
\item Founder and member, Queer Union for (Bio)statistician Inclusion and Community affinity group (2022--2023)
\item Graduate representative, Undergraduate statistics curriculum revamp committee (2021--22)
\end{itemize}

\textbf{Broader Community}
\begin{itemize}
\item Anonymous peer reviewer ({\it PlosOne, Biomedicine Hub, Cogent Social Sciences, Advances in Data Science and Classification, Journal of Statistics and Data Science Education})
\item Judge, Undergraduate Statistics Project Competition (December 2023 cycle)
\item Workflow chair, International Conference on Machine Learning (2021)
\end{itemize}




\end{rSection}


\begin{rSection}{Honors and Awards}

\textbf{Distinguished Dissertation Award} {\it (The Classification Society)} \hfill {2024}
\\ \textbf{Best Poster Award} {\it (International Society for Bayesian Analysis World Meeting)} \hfill {2024}
\\ \textbf{Travel Award} {\it (International Society for Bayesian Analysis)} \hfill {2022, 2024}
\\ \textbf{Scholar Award} {\it (NeurIPS)} \hfill {2022}
\\ \textbf{Dorothy M. Gilford Teaching Award} {\it (University of Washington)} \hfill {2021}

\end{rSection}


\begin{rSection}{Industry Experience}
\textbf{Boeing Research and Technology}\hfill{2019--2020}
\\{\it Applied Statistics Intern}

\textbf{Deloitte LLC}\hfill{2017--2018}
\\{\it Analytics Consultant}

\end{rSection}

\begin{rSection}{Software}

\textbf{rankclust: Fit a Bayesian, Rank-Clustered Bradley-Terry-Luce Model to Ordinal Comparison Data.} R package available on Github. Vignettes available \href{https://pearce790.github.io/rankclust/}{here}.

\textbf{rankrate: Statistical Tools for Preference Learning with Rankings and Ratings.} R package available on CRAN. Vignettes available \href{https://pearce790.github.io/rankrate/}{here}.

\textbf{Peer Review with Rankings and Ratings.} R Shiny \href{https://pearce790.shinyapps.io/rankrate_PeerReview/}{application}.

\end{rSection}

\begin{rSection}{Professional Affiliations}

American Statistical Association
\\ International Society for Bayesian Analysis
\\ Psychometric Society

\end{rSection}



\end{document}
